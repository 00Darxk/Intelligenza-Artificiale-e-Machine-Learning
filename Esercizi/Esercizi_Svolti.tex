\documentclass{article}

\usepackage{cancel}
\usepackage{amsmath}
\usepackage[includehead,nomarginpar]{geometry}
\usepackage{graphicx}
\usepackage{amsfonts} 
\usepackage{verbatim}
\usepackage{mathrsfs}  
\usepackage{lmodern}
\usepackage{braket}
\usepackage{bookmark}
\usepackage{fancyhdr}
\usepackage{romanbarpagenumber}
\usepackage{minted}
%\usepackage{subfig}
\usepackage[italian]{babel}
\usepackage{float}
%\usepackage{wrapfig}
%\usepackage[export]{adjustbox}
\usepackage{graphicx}
\allowdisplaybreaks

\setlength{\headheight}{12.0pt}
\addtolength{\topmargin}{-12.0pt}
\graphicspath{ {../Immagini/} }

\hypersetup{
    colorlinks=true,
    linkcolor=black,
}

\newsavebox{\tempbox} %{\raisebox{\dimexpr.5\ht\tempbox-.5\height\relax}}

\makeatother

\numberwithin{equation}{section}
\newcommand{\tageq}{\tag{\stepcounter{equation}\theequation}}
\newcommand{\vv}[1]{\verb{\text{#1}}}
\AtBeginDocument{%
  \renewcommand{\figurename}{Fig.}
}
\fancypagestyle{link}{\fancyhf{}\renewcommand{\headrulewidth}{0pt}\fancyfoot[C]{Sorgente del file LaTeX disponibile al seguente link: \url{https://github.com/00Darxk/Intelligenza-Artificiale-e-Machine-Learning}}}

\begin{document}

\title{%
    \textbf{Intelligenza Artificiale e Machine Learning}  \\ 
    \large Esercizi Svolti di Intelligenza Artificiale e Machine Learning \\
    \textit{Anno Accademico: 2024/25}}
\author{\textit{Giacomo Sturm}}
\date{\textit{Dipartimento di Ingegneria Civile, Informatica e delle Tecnologie Aeronautiche \\
Università degli Studi ``Roma Tre"}}

\maketitle
\thispagestyle{link}

\clearpage


\pagestyle{fancy}
\fancyhead{}\fancyfoot{}
\fancyhead[C]{\textit{Intelligenza Artificiale e Machine Learning - Università degli Studi ``Roma Tre"}}
\fancyfoot[C]{\thepage}
\pagenumbering{Roman}

\tableofcontents

\clearpage
\pagenumbering{arabic}

\section{Esercitazione del 13 Novembre}

\subsection{Esercizio 1}

Dato il seguente Spazio degli Stati: $s_0,s_1,s_2,s_3,s_4,s_5$, con stato iniziale $s_0$, e stato obiettivo $s_5$. 
Questi stati sono collegati da certi operatori di costo:
\begin{itemize}
    \item $F$: costo 3. $F(s_0)=s_2, F(s_1)=s_3, F(s_4)=s_5$;
    \item $G$: costo 4. $G(s_3)=s_2, G(s_3)=s_4, G(s_4)=s_3$;
    \item $H$: costo 5; $H(s_0)=s_1, H(s_2)=s_0, H(s_2)=s_4, H(s_4)=s_2, H(s_5)=s_1$;
    \item $I$: costo 6; $I(s_3)=s_5$.
\end{itemize}
Nel modo seguente: 

\begin{figure}[H]%
    \centering%
    \includegraphics[scale=1.2]{grafo_esercitazione_1.pdf}%
\end{figure}
E la seguente funzione euristica: $h(s_0)=10$, $h(s_1)=6$, $h(s_2)=7$, $h(s_3)=4$, $h(s_4)=5$, $h(s_5)=0$. 

\begin{enumerate}
    \item Eseguire l'algoritmo A* con modalità tree-search, fino alla terminazione disegnando l'albero di ricerca, riportando per ogni passo il 
contenuto della frontiera, il nodo scelto per l'espansione ed i nodi generati;
    \item Riportare la soluzione trovata dell'algoritmo, indicando se tale soluzione è o non è quella ottima e indicando se la funzione euristica 
$h$ rispetta la condizione di ammissibilità (motivare la risposta). 
\end{enumerate}

\subsubsection*{Domanda 1}

Bisogna specificare i vari passi ottenuti dall'algoritmo A*, specificando la frontiera i nodi contenuti ed espansi ed il nodo successivo 
da espandere.

\begin{enumerate}
    \item Nella frontiera è presente solamente lo stato iniziale $[s_0]$;
    \item Si espande lo stato iniziale $s_0$, nella frontiera si ha $[s_2,s_1]$, dove $f(s_2)=3+7$ e $f(s_1)=5+6$;
    \item Si espande il nodo $s_2$, con $f(s_2)=10$, per cui nella frontiera entra il nodo $s_4$ e $s_0$: $[s_1,s_4, s_0]$, con $f(s_1)=5+6$, $f(s_4)=8+5$ e $f(s_0)=8+10$;
    \item Si espande il nodo $s_1$, con $f(s_1)=12$, per cui nella frontiera entra il nodo $s_3$: $[s_3, s_4, s_0]$, con $f(s_3)=8+4$, $f(s_4)=8+5$ e $f(s_0)=8+10$;
    \item Si espande il nodo $s_3$, con $f(s_3)=12$, per cui nella frontiera entra il nodo $s_2$, $s_4$ ed $s_5$: $[s_4(13), s_5, s_4(17), s_0, s_2]$, con $f(s_4)=8+5$, $f(s_5)=14+0$, $f(s_4)=12+5$, $f(s_0)=8+10$ ed $f(s_2)=12+7$;
    \item Si espande il nodo $s_4$ con $f(s_4)=13$, per cui nella frontiera entra il nodo $s_2$, $s_3$ ed $s_5$: $[s_5(11), s_5(14), s_3, s_4, s_0, s_2(19), s_2(20)]$, con $f(s_5)=11+0$, $f(s_5)=14+0$, $f(s_3)=12+4$, $f(s_4)=12+5$, $f(s_0)=8+10$, $f(s_2)=12+7$, $f(s_2)=13+7$;
    \item Si espande il nodo $s_5$, con $f(s_5)=11$. Questo nodo è il nodo obiettivo, quindi l'algoritmo termina. 
\end{enumerate}

I nodi rossi sono quelli nella frontiera, i nodi blu sono i nodi nella frontiera da espandere:

\begin{figure}[H]%
    \centering%
    \includegraphics[trim={3.9cm 0 0 0}, scale=0.95]{albero_esercitazione_1.pdf}%
\end{figure}

\subsubsection*{Domanda 2}

La soluzione dell'algoritmo è $s_0,s_2,s_4,s_5$, di costo 11:
\begin{figure}[H]%
    \centering%
    \includegraphics[trim={4cm 0 0 0}]{soluzione_esercitazione_1.pdf}%
\end{figure}

La funzione euristica $h$ utilizzata è maggiore del costo effettivo per il nodo $s_4$: 
\begin{equation}
    h(s_4)=5>3=h^*(s_4)
\end{equation}

Quindi non è una funzione ammissibile. La soluzione trovata 
non è garantito sia la soluzione ottima. 

\subsection{Esercizio 2}

Scrivere il codice Python per la gestione di una lista concatenata ordinata, avvalendosi delle classi. \\
\begin{minted}{python}
    class Node:
        def __init__(self, data, next)
            self.__data = data
            self.__next = next

    class List:
        def __init__(self):
            self.__head = None
            self.__tail = None

        def add(self, newNode):            
            if self.__head == None or self.__head.data > newNode.data:
                if self.__head == None:
                    self.__tail = newNode
                newNode.next = self.__head
                self.__head = newNode

            elif self.__tail.data < newNode.data:
                if self.__tail != None:
                    self.__tail.next = newNode
                if self.__head == None:
                    self.__head = NewNode
                self.__tail = newNode

            else:            
                p = self.__head
                while p.next != None and p.data < newNode.data:
                    p = p.next
                newNode.next = p.next
                p.next = newNode
\end{minted}

\subsection{Esercizio 3}

Illustrare nel dettaglio l'algoritmo ``Greedy'', presentando il codice Python, opportunamente commentato per il problema della ricerca di un 
itinerario. \\
Per realizzare l'algoritmo sono necessarie le classi per gli stati, individuati solamente dal loro nome nello spazio degli stati, 
ed i nodi dell'albero di ricerca:
\begin{minted}{python}
    # nodo dell'albero di ricerca: contenente lo stato, il nodo genitore
    # ed il valore della funzione euristica per lo stato
    class Node:
        def __init__(self, state, parent, h):
            self.state = state
            self.parent = parent
            self.h = h
        
        # funzione per stampare la soluzione
        def printPath(self):
            if self.parent != None:
                self.parent.printPath()
            print("->", self.state.name)
    
    # stato del problema
    class State:
        # crea uno stato, definito solamente dal nome
        # se non viene specificato crea lo stato iniziale
        def __init__(self, name = None):
            if name == None:
                self.name = self.getInitialState()
            else:
                self.name = name

        # restituisce lo stato iniziale definito a priori
        def getInitialState(self):
            initialState = statoIniziale
            return initialState

        # ottiene i successori dal dizionario 'connections'
        def successorFunction(self):
            return connections[self.name]

        # verifica se è uno stato obiettivo definito a priori
        def checkGoalState(self):
            return self.name == statoObiettivo
\end{minted}
La funzione euristica ed i nodi successori per un determinato stato si possono implementare 
allo stesso modo tramite un dizionario, dove la chiave è il nome dello stato, ed il valore 
è o il valore dell'euristica o la lista degli stati successori:
\begin{minted}{python}
    # dizionario dei successori: 
    connections['A'] = ['B', 'C']
    connections['B'] = ['C']
    connections['C'] = ['B', 'D']
    connections['D'] = ['B', 'C', 'E']
    connections['E'] = ['A', 'C']
    # dizionario delle euristiche
    h['A'] = 10
    h['B'] = 9
    h['C'] = 6
    h['D'] = 3
    h['E'] = 0
\end{minted}

L'algoritmo inizializza la frontiera con coda di priorità, contenente solamente l'elemento 
corrispondente alla radice. In seguito continua ad iterare fino a quando non svuota la 
frontiera, rimuovendo il primo elemento in frontiera, rappresentati come tuple euristica e 
nodo. In questo modo si possono ordinare in base al loro valore di euristica. Se il nodo 
estratto è un obiettivo, termina l'algoritmo e stampa la soluzione, altrimenti viene espanso ed i suoi 
figli vengono inseriti all'interno della frontiera. 
\begin{minted}{python}
    import queue 

    def Greedy_Best_First():
        # crea la frontiera con una priority queue
        fringe = queue.PriorityQueue() 

        # crea lo stato iniziale
        initialState = State()

        # crea la radice dell'albero di ricerca e la aggiunge alla frontiera
        root = Node(initialState, None, h[initialState.name])
        fringe.put((root.h, root))

        while not fringe.empty():
            # si itera prendendo il primo elemento della frontiera
            # fino all'esaurimento dei suoi elementi
            (currentH, currentNode) = fringe.get()

            # si controlla se corrisponde ad uno stato obiettivo
            if currentNode.state.checkGoalState():
                currentNode.state.printPath()
                break
            # altrimenti si espande e si aggiungono i suoi figli alla frontiera
            else: 
                childStates = currentNode.state.successorFunction()
                # crea gli elementi corrispondenti per l'aggiunta in frontiera
                for childState in childStates:
                    childNode = Node(State(childState), currentNode, 
                                        h[State(childState).name])
                    fringe.put((childNode.h, childNode))
\end{minted}

\clearpage

\section{Esercitazione del 25 Novembre}

\subsection{Esercizio 1}

Dati i seguenti stati $s_0, s_1,s_2,s_3,s_4,s_5$, si hanno le seguenti operazioni:
\begin{itemize}
    \item $F$ di costo 2: $F(s_0)=s_2$;
    \item $G$ di costo 3: $G(s_0)=s_3$, $G(s_2)=s_4$, $G(s_4)=s_2$;
    \item $H$ di costo 4: $H(s_0)=s_1$, $H(s_3)=s_2$, $H(s_3)=s_4$, $H(s_4)=s_3$;
    \item $I$ di costo 5: $I(s_5)=s_1$;
    \item $L$ di costo 7: $L(s_3)=s_5$;
\end{itemize}

Con la seguente funzione euristica: $h(s_0)=10$, $h(s_1)=7$, $h(s_2)=7$, $h(s_3)=5$, $h(s_4)=4$ e $h(s_5)=0$. 

\begin{enumerate}
    \item Disegnare lo spazio degli stati;
    \item Eseguire l'algoritmo A* con modalità graph-search, fino alla terminazione disegnando l'albero di ricerca, riportando per ogni passo il 
contenuto della frontiera e della close, il nodo scelto per l'espansione ed i nodi generati;
    \item Riportare la soluzione trovata dell'algoritmo, indicando se tale soluzione è o non è quella ottima e indicando se la funzione euristica 
$h$ rispetta la condizione di ammissibilità (motivare la risposta). 
\end{enumerate}

\subsubsection*{Domanda 1}

Si ha lo spazio degli stati:
\begin{figure}[H]%
    \centering%
    \includegraphics{spazio_stati_25-11-24.pdf}%
\end{figure}

\subsubsection*{Domanda 2}


Si effettua l'algoritmo A* in modalità graph-search, quindi quando viene espanso un nodo, il suo stato corrispondente viene inserito nella lista close e quando viene scelto per l'espansione viene scartato e si passa al passo successivo. I nodi relativi ai nodi già visitati dato che non verranno sicuramente espansi non vengono inseriti nella frontiera quando vengono generati da uno stato genitore. Si descrive il processo dell'algoritmo:
\begin{enumerate}
    \item La close è vuota, nella frontiera è presente solo il nodo corrispondente allo stato iniziale $[s_0]$, e la close è vuota $[]$;
    \item Viene estratto dalla frontiera ed espanso lo stato $s_0$ ed i suoi figli $s_1$, $s_2$ ed $s_3$ vengono inseriti nella frontiera $[s_3,s_2,s_1]$ con $f(s_3)=8$, $f(s_2)=9$ e $f(s_1)=11$. La close contiene lo stato iniziale $[s_0$];
    \item Viene espanso il nodo $s_3$ e vengono inseriti nella frontiera i suoi figli $s_2$, $s_4$ e $s_5$. Bisogna distinguere i due nodi associati entrambi allo stato $s_2$, tramite il valore della loro funzione di valutazione: $[s_2(9),s_5,s_1,s_4,s_2(14)]$ con $f(s_2)=8$, $f(s_5)=10$, $f(s_1)=11$, $f(s_4)=11$ e $f(s_2)=14$. Mentre la close contiene i nodi $[s_0,s_3]$;
    \item Viene espanso il nodo $s_2$ e si inserisce il suo nodo figlio nella frontiera, specificando il valore per distinguerlo dall'altro nodo associato allo stato $s_4$. Poiché lo stato $s_2$ è già stato visitato si potrebbe rimuovere dalla frontiera, poiché anche se venisse scelto non verrebbe espanso. La frontiera contiene i nodi $[s_4(9),s_5,s_1,s_4(11),\cancel{s_2}]$, con $f(s_4)=9$, $f(s_5)=10$, $f(s_1)=11$, $f(s_4)=11$ e $f(s_2)=14$. La close contiene i nodi $[s_0,s_3,s_2]$;
    \item Viene espanso il nodo $s_4$, i suoi nodi corrispondono a stati presenti nella close, quindi non vengono inseriti nella frontiera. La frontiera è quindi $[s_5,s_1,\cancel{s_4},\cancel{s_2}]$. La close contiene i nodi $[s_0,s_3,s_2,s_4]$;
    \item Viene estratto il nodo $s_5$, questo rappresenta lo stato obiettivo, quindi l'algoritmo termina. 
\end{enumerate}

In nero sono indicati i nodi espansi, quindi all'interno della close, in rosso i nodi all'interno della frontiera ed in blu i nodi scelti per l'espansione nel passo corrente, l'algoritmo termina quando viene scelto il nodo associato allo stato $s_5$ per l'espansione:

\begin{figure}[H]%
    \centering
    \includegraphics[trim={3.9cm 0 0 0},scale=0.9]{albero_ricerca_25-11-24.pdf}%
\end{figure}

\subsubsection*{Domanda 3}

La soluzione individuata è $s_0\rightarrow s_3\rightarrow s_5$, di costo $10$. Queste rappresenta la soluzione ottima. La funzione euristica è ammissibile, poiché per ogni stato $s_i$ la funzione euristica $h(s_i)$ è minore, o uguale per gli stati $s_0$ e $s_5$, al costo effettivo del cammino dal nodo $s_i$ al nodo obiettivo $s_5$. 

%% TODO resto esercitazione rec 25/11/24

\end{document}