\documentclass{article}

\usepackage{cancel}
\usepackage{amsmath}
\usepackage[includehead,nomarginpar]{geometry}
\usepackage{graphicx}
\usepackage{amsfonts} 
\usepackage{verbatim}
\usepackage{mathrsfs}  
\usepackage{lmodern}
\usepackage{braket}
\usepackage{bookmark}
\usepackage{fancyhdr}
\usepackage{romanbarpagenumber}
\usepackage{minted}
%\usepackage{subfig}
\usepackage[italian]{babel}
%\usepackage{float}
%\usepackage{wrapfig}
%\usepackage[export]{adjustbox}
\allowdisplaybreaks

\setlength{\headheight}{12.0pt}
\addtolength{\topmargin}{-12.0pt}
\graphicspath{ {./Immagini/} }

\hypersetup{
    colorlinks=true,
    linkcolor=black,
}

\newsavebox{\tempbox} %{\raisebox{\dimexpr.5\ht\tempbox-.5\height\relax}}


\makeatother

\numberwithin{equation}{subsection}
\newcommand{\tageq}{\tag{\stepcounter{equation}\theequation}}
\AtBeginDocument{%
  \renewcommand{\figurename}{Fig.}
}
\fancypagestyle{link}{\fancyhf{}\renewcommand{\headrulewidth}{0pt}\fancyfoot[C]{Sorgente del file LaTeX disponibile al seguente link: \url{https://github.com/00Darxk/Intelligenza-Artificiale-e-Machine-Learning}}}

\begin{document}

\title{%
    \textbf{Intelligenza Artificiale e Machine Learning}  \\ 
    \large Appunti delle Lezioni di Intelligenza Artificiale e Machine Learning \\
    \textit{Anno Accademico: 2024/25}}
\author{\textit{Giacomo Sturm}}
\date{\textit{Dipartimento di Ingegneria Civile, Informatica e delle Tecnologie Aeronautiche \\
Università degli Studi ``Roma Tre"}}

\maketitle
\thispagestyle{link}

\clearpage


\pagestyle{fancy}
\fancyhead{}\fancyfoot{}
\fancyhead[C]{\textit{Intelligenza Artificiale e Machine Learning - Università degli Studi ``Roma Tre"}}
\fancyfoot[C]{\thepage}
\pagenumbering{Roman}

\tableofcontents

\clearpage
\pagenumbering{arabic}

\section{Introduzione}

\subsection{Storia dell'Intelligenza Artificiale}

\clearpage

\section{Risoluzione dei Problemi e Ricerca}

Si definisce un agente risolutore di problemi un agente con uno specifico obiettivo da 
raggiungere e che deve identificare una sequenza di azioni per raggiungerlo. 

Bisogna determinare l'obiettivo, un insieme degli stati del mondo dove ci si trova, che 
si vuole raggiungere. 
Inoltre bisogna formulare il problema, ovvero le azioni e gli stati considerati 
dall'agente. 


Un agente che ha a disposizioni diverse opzioni immediate di valore sconosciuto, può decidere 
quale scegliere la sua azione analizzando le diverse possibili sequenze di azioni, che portano 
a stati di valore conosciuto, per scegliere la sequenza di costo migliore. 

Questo processo di selezione viene definito ricerca, un algoritmo di ricerca quindi  
prende un problema e restituisce una soluzione costituita da una sequenza di azioni. 

Nella formulazione si definisce uno stato iniziale e dell'obiettivo, come un insieme di stati 
e si definiscono le azioni come transizioni tra stati. Dopo aver trovato la sequenza di azioni 
corrispondente alla soluzione, la esegue. 


Si possono distinguere due tipi di problemi, i problemi giocattolo o ``toy problems'', sono 
ideati come illustrazione o esercitazione dei metodi risolutivi. Rappresentano delle 
astrazioni, anche semplificate, dei problemi del mondo reale in generale più difficili a cui 
si è effettivamente interessati. 

I problemi del mondo reale possono essere la configurazione VLSI, la navigazione dei robot, 
la sequenza di montaggio, la ricerca dell'itinerario e generali problemi di viaggio come il 
commesso viaggiatore. 



Si possono utilizzare due tipi diversi di formulazione di un problema. Si può partire da uno 
stato vuoto, ed utilizzando operatori si può estendere progressivamente la descrizione 
dello stato. 
Invece nella formulazione a stato completo consiste nel partire da uno stato iniziale completo, 
ed ogni operatore altera questo stato per cercare una soluzione. 


Lo spazio degli stati è un grafo che rappresenta tutti i possibili stati, come nodi, collegati 
tra archi che rappresentano le possibili azioni. Il problema consiste nel trovare un 
percorso in questo spazio degli stati, dallo stato iniziale ad uno dei possibili stati 
soluzione. L'algoritmo deve decidere ad ogni stato quale azione prendere e quindi a quale nodo 
del grafo spostarsi. 
Per definire il costo della soluzione, si considera il costo del cammino delle azioni 
intraprese dall'agente. 

Per definire formalmente un problema, sono necessarie quattro componenti. Uno stato iniziale 
in cui si trova l'agente. Una descrizione delle azioni possibili, questa può utilizzare una 
funzione successore che dato uno stato restituisce i suoi possibili successori. Tramite la 
funzione successore e lo stato iniziale si può costruire lo spazio degli stati. Oppure si 
può utilizzare un insieme di operatori. Un test obiettivo per determinare se un particolare 
stato è uno stato obiettivo. Come ultimo componente è necessaria una funzione di costo del 
cammino per determinare il costo di una data soluzione, assegnando un valore numerico ad 
ogni cammino. 

Il tipo di dato problema è rappresentato da questi quattro componenti, le istanze di questo 
tipo di dato rappresentano gli input degli algoritmi di ricerca. 


Il mondo reale è estremamente complesso, e lo spazio degli stati deve essere creato mediante un 
processo di astrazione. In questo processo, lo stato astratto rappresenta un insieme di 
stati reali più complessi. Analogamente per le azioni astratte, queste rappresentano combinazioni 
di azioni reali. Nei problemi giocattolo non è necessario effettuare questo processo di astrazione, 
poiché rappresenta un problema semplice. 


Per individuare queste possibili sequenze di azioni l'algoritmo si può costruire un albero di 
ricerca, dove il nodo radice corrisponde allo stato iniziale, ed i rami rappresentano 
le azioni possibili, ed i nodi figli rappresentano gli stati successori di un certo stato. 
Tuttavia i nodi dell'albero di ricerca e gli stati dello spazio degli stati sono differenti, poiché 
è possibile che più nodi condividano lo stesso stato, mentre ogni stato nello spazio è univoco. 
Generalmente si vuole evitare di ripetere stati all'interno di un cammino, questo rappresenta il 
problema degli stati ripetuti, e sarà analizzato successivamente. 

Ad ogni nodo si possono inserire altre informazioni utili, oltre allo stato, un riferimento al 
genitore, l'operatore che ha generato lo stato, la profondità, il costo del cammino parziale 
fino a questo stato, etc. 
\begin{minted}{python}
    NODO = <stato, genitore, operatore, profondità, costo parziale, ...>
\end{minted}

Il processo di ricerca comporta la stessa sequenza di azioni. Deve scegliere tra le foglie 
dell'albero corrente un nodo da ``espandere'', secondo un certo criterio o strategia. 
In seguito bisogna determinare se questo nodo rappresenta un'obiettivo del problema, 
altrimenti vengono generati i suoi nodi figli, ed i corrispondenti stati successori, e tutte 
le componenti dei nodi figli. 
La collezione dei nodi in attesa di essere espansi viene chiamata in vari modi: confine, 
frontiera, frangia o lista aperta. 

\subsection{Algoritmo di Ricerca Generale: Tree Search}

Un algoritmo generale di ricerca può essere chiamato \color{red}\verb|TREE-SEARCH|\color{black}. Quest'algoritmo 
prende un problema ed una strategia come input e restituisce una soluzione oppure un fallimento. 
Viene realizzato semplicemente ad un ciclo che ripete le operazioni precedentemente descritte, 
fino a quando non identifica una soluzione o viene sollevato un problema, e quindi restituisce 
un fallimento. Il primo passo è la generazione dell'albero di ricerca del problema, se la frontiera è 
vuota, ovvero non esistono nodi candidati per l'espansione viene riportato un fallimento, poiché 
non è stato ancora trovato uno stato obiettivo. Se si arriva ad un nodo corrispondente ad uno 
stato obiettivo viene restituita rappresenta la sequenza di nodi ottenuta come soluzione. 

La frontiera può contenere un nodo relativo ad uno stato obiettivo, ma l'algoritmo non 
termina fino a quando non viene scelto per essere espanso. 


Una strategia di ricerca rappresenta un criterio per decidere quale nodo da espandere. 
Può essere definita come una funzione per la scelta di un elemento tra un insieme di nodi, la 
frontiera. Oppure si può considerare come una funzione di inserimento di un elemento in una 
sequenza. Se già nella fase di inserimento si analizza tramite una metrica il valore di ognuno 
di questi stati, allora l'elemento in prima posizione in questa struttura dati rappresenta il 
nodo migliore. 

\subsubsection{Operazioni su Frontiera, Nodi e Problemi}

La frontiera viene implementata con una struttura dati chiamata coda, ma non necessariamente 
segue la disciplina FIFO. Su questa coda sono definite una serie di operazioni:
\begin{itemize}
    \item \color{magenta}\verb|MAKE-QUEUE|\color{black}: prende come input un nodo \verb|n| e restituisce una coda \verb|q|, realizza una coda contenente solo il nodo \verb|n|;
    \item \color{magenta}\verb|EMPTY|\color{black}: prende come input una coda \verb|q| e restituisce un booleano, verifica se la coda è vuota;
    \item \color{magenta}\verb|REMOVE-FRONT|\color{black}: prende come input una coda \verb|q| e restituisce il primo nodo \verb|n| della lista; 
    \item \color{magenta}\verb|QUEUING-FN|\color{black}: prende come input una coda \verb|q| ed una lista di nodi \verb|n|, e restituisce la coda con aggiunti tutti questi nodi. 
\end{itemize}
L'ultima funzione si utilizza quando si producono tutti i nodi successori e si vogliono aggiungere 
alla lista. 
Questa funzione non inserisce generalmente in coda, ma dipende dalla strategia di ricerca 
utilizzata. 

Assumendo che esistano le operazioni sul tipo di dato problema, e sul tipo di dato nodo. Le 
operazioni sul tipo di dato problema sono:
\begin{itemize}
    \item \color{magenta}\verb|INITIAL-STATE|\color{black}: prende come input un problema \verb|p| e restituisce lo stato iniziale del problema \verb|n|;
    \item \color{magenta}\verb |GOAL-TEST|\color{black}: prende come input un problema \verb|p| ed uno stato \verb|n| e verifica se questo rappresenta una soluzione, restituendo un booleano;
    \item \color{magenta}\verb|OPERATORS|\color{black}: prende come input un problema \verb|p| e restituisce una lista con tutti gli operatori del problema \verb|ops|. Ogni operatore \verb|op| applicato ad uno stato \verb|n| restituisce una lista di stati \verb|ss|. 
\end{itemize}

Le operazioni sul tipo di dato nodo sono:
\begin{itemize}
    \item \color{magenta}{\verb|MAKE-NODE|}\color{black}: prende come input uno stato \verb|s| e costruisce un nodo su di esso \verb|n|;
    \item \color{magenta}\verb|STATE|\color{black}: prende come input un nodo \verb|n| e ne restituisce lo stato contenuto \verb|s|;
    \item \color{magenta}\verb|EXPAND|\color{black}: prende come input un nodo \verb|n| ed una lista di operatori \verb|ops| e restituisce una lista di nodi successori \verb|ns|. 
\end{itemize}

\subsubsection{Implementazione}

Date queste operazioni, si può rappresentare in pseudocodice l'algoritmo di \verb|TREE-SEARCH| in modo più 
semplice:
\begin{minted}[escapeinside=||]{python}
    function |\textcolor{red}{TREE-SEARCH}|(problem) returns a solution or failure

    fringe <- |\textcolor{magenta}{MAKE-QUEUE}|(|\textcolor{magenta}{MAKE-NODE}|(|\textcolor{magenta}{INITIAL-STATE}|(problem)))
    loop do
        if |\textcolor{magenta}{EMPTY}|(fringe) then return failure
        node <- |\textcolor{magenta}{REMOVE-FRONT}|(fringe)
        if |\textcolor{magenta}{GOAL-TEST}|(problem, |\textcolor{magenta}{STATE}|(node)) then return |\textcolor{magenta}{SOLUTION}|(node)
        fringe <- |\textcolor{magenta}{QUEUING-FN}|(fringe, |\textcolor{magenta}{EXPAND}|(node, |\textcolor{magenta}{OPERATOR}|(problem)))
    end
\end{minted}

In questa variazione non viene conservato l'intero albero, ma solamente la coda con i nodi della 
frontiera. 

\subsection{Criteri di Valutazione}

Per valutare questi algoritmi oltre alla complessità temporale e spaziale, si utilizzano 
altre due criteri, la completezza e l'ottimalità. Un'algoritmo si definisce completo, se 
quando esiste una soluzione è garantito sia in grado di trovarla. Un algoritmo si dice 
ottimo se dato un problema con diverse soluzioni, individua sempre la migliore, quella a costo 
minimo. 
La complessità dell'algoritmo dipende dal fattore di ramificazione $b$ dello spazio degli stati e dalla profondità $d$ 
della soluzione più superficiale. Il fattore di ramificazione $b$ rappresenta il massimo numero di 
figli che un nodo può avere. Mentre la profondità $d$ è la minima lunghezza di un cammino dal 
nodo iniziale alla radice. 

\subsection{Ricerca non Informata o Cieca}
Questi algoritmi non sono molto efficienti in generale, ma sono utili per comprendere il 
comportamento gli algoritmi di ricerca informata o di euristica, che si avvalgono della 
conoscenza sul domino dello spazio degli stati e dalla creazione dell'albero di 
ricerca per scegliere il percorso più promettente. Nel caso medio quest'ultimi sono certamente 
più efficienti degli algoritmi trattati in questa sezione. 

\subsubsection{Algotimo di Ricerca in Ampiezza: Breadth First Search (BFS)}

Nella ricerca in ampiezza si espande il nodo radice, e si espandono i nodi generati dalla 
radice, e si ripete per ogni nodo successore. Per implementare un algoritmo che utilizza una 
strategia di ricerca non informata in ampiezza, la ``Queueing Function'' inserisce i nodi 
appena generati in coda. Questa funzione quindi rappresenta sempre un inserimento in coda 
e si può chiamare ``Enqueue at the End'': \color{magenta}\verb|ENQUEUE-AT-END|\color{black}. 

Un algoritmo che utilizza questo tipo di strategia viene chiamato ``Breadth First Search'' o 
in ampiezza, e si può implementare in modo analogo all'algoritmo di ricerca generale trattato 
precedentemente:
\begin{minted}[escapeinside=||]{python}
    function |\textcolor{red}{BREADTH-FIRST-SEARCH}|(problem) returns a solution or failure
    
    fringe <- |\textcolor{magenta}{MAKE-QUEUE}|(|\textcolor{magenta}{MAKE-NODE}|(|\textcolor{magenta}{INITIAL-STATE}|(problem)))
    loop do
        if |\textcolor{magenta}{EMPTY}|(fringe) then return failure
        node <- |\textcolor{magenta}{REMOVE-FRONT}|(fringe)
        if |\textcolor{magenta}{GOAL-TEST}|(problem, |\textcolor{magenta}{STATE}|(node)) then return |\textcolor{magenta}{SOLUTION}|(node)
        fringe <- |\textcolor{magenta}{ENQUEUE-AT-END}|(fringe, |\textcolor{magenta}{EXPAND}|(node, |\textcolor{magenta}{OPERATOR}|(problem)))
    end
\end{minted}

In questo approccio, tutti i nodi di profondità $d$ vengono espansi prima dei nodi di profondità 
$d+1$. Rappresenta una strategia sistematica, ma permette di individuare solamente i nodi 
obiettivi più superficiali, non è garantito che questo rappresenta la soluzione ottima. Questo 
algoritmo è quindi completo, ma non è ottimo. 
Invece è ottimale se il costo del cammino $g(n)$ è una funzione monotona non decrescente della 
profondità del nodo $p(n)$:
\begin{gather*}
    p(n)<p(n)\implies g(n)\leq g(m)\\
    p(n)=p(m)\implies g(n)=g(m)
\end{gather*} 

Ovvero se due nodi $n$ ed $m$ sono a profondità diverse, dove il nodo $m$ è a profondità maggiore, 
il costo del cammino dalla radice al nodo $n$ è al massimo uguale al costo del cammino dalla radice 
al nodo $m$. Inoltre se i nodi sono alla stessa profondità, allora i costi dei loro cammini dalla radice 
sono uguali. 


Utilizzando questo algoritmo, bisogna generare un numero di nodi, prima di trovare una soluzione,  
almeno pari a tutti i nodi precedenti al nodo soluzione. Nel caso peggiore questo nodo è 
l'ultimo nodo espanso alla profondità $d$, e per ogni nodo vengono generati esattamente $b$ 
figli, quindi bisogna espandere al massimo un numero di nodi $N$ pari a:
\begin{gather*}
    N=\left(\displaystyle\sum_{i=1}^{d+1}b^i\right)-b
\end{gather*}

Supponendo che ogni generazione rappresenta un'operazione semplice allora la complessità 
temporale di questa ricerca è $O(N)=O(b^d)$. La complessità temporale è analogamente 
$O(b^d)$ poiché bisogna memorizzare tutte le foglie generate. 

\subsubsection{Algoritmo di Ricerca Guidata dal Costo: Dijkstra}

Modificando la ricerca in ampiezza espandendo il nodo della frontiera di costo più basso, si può 
aumentare l'efficienza dell'algoritmo precedente. 

Si definisce con $g(n)$ il costo del cammino dalla radice al nodo $n$. Questo valore viene 
salvato nella struttura dati nodo, e viene scelto il nodo di costo $g(n)$ minore per essere 
espanso. Se il costo del cammino corrisponde alla funzione di profondità, si ha la ricerca 
in ampiezza. 
Questo algoritmo è completo e ottimale quando il costo di ogni step è sempre 
maggiore o uguale ad una costante positiva $\varepsilon$. Per cui è garantito che non attraversi 
costantemente lo stesso cammino, senza espandere altri nodi di profondità minore. 

I costi di ogni nodo vengono salvati in un campo etichetta nella struttura dati nodo. 
Dalla frontiera si estrae sempre il nodo a costo minore, questa collezione viene quindi 
ordinata in base al costo delle etichette di ogni nodo. 

\subsubsection{Algoritmo di Ricerca in Profondità: Depth First Search (DFS)}

La ricerca in profondità consiste nell'espansione del nodo più profondo, dopo aver espanso 
la radice. 
Per implementare questa funzione, si utilizza una queueing function che inserisce i nodi 
appena espansi all'inizio della lista, con una ``Enqueue at the Front'': \color{magenta}\verb|ENQUEUE-AT-FRONT|\color{black}:

\begin{minted}[escapeinside=||]{python}
    function |\textcolor{red}{BREADTH-FIRST-SEARCH}|(problem) returns a solution or failure
    
    fringe <- |\textcolor{magenta}{MAKE-QUEUE}|(|\textcolor{magenta}{MAKE-NODE}|(|\textcolor{magenta}{INITIAL-STATE}|(problem)))
    loop do
        if |\textcolor{magenta}{EMPTY}|(fringe) then return failure
        node <- |\textcolor{magenta}{REMOVE-FRONT}|(fringe)
        if |\textcolor{magenta}{GOAL-TEST}|(problem, |\textcolor{magenta}{STATE}|(node)) then return |\textcolor{magenta}{SOLUTION}|(node)
        fringe <- |\textcolor{magenta}{ENQUEUE-AT-FRONT}|(fringe, |\textcolor{magenta}{EXPAND}|(node, |\textcolor{magenta}{OPERATOR}|(problem)))
    end
\end{minted}

Questa funzione si può implementare mediante una funzione ricorsiva, dove viene 
passata una versione del problema, dove i nodi appena generati rappresentano i nuovi 
nodi radice. Quindi per ogni espansione vengono generati al massimo $b$ sotto-problemi, risolti 
dallo stesso algoritmo. La lista dei nodi da visitare viene conservata implicitamente nello 
stack dei record di attivazione delle varie chiamate ricorsive. Quando si raggiunge un nodo 
foglia non obiettivo, si effettua il backtracking, ovvero si risale l'albero fino a trovare il 
nodo a profondità maggiore non ancora espanso su cui è possibile effettuare una scelta. 

Questa ricerca non è né completa né ottimale, ha una complessità temporale di $O(b^m)$, dove 
$m$ rappresenta la profondità massima dell'albero di ricerca. Se una soluzione è presente a 
profondità minore nel sotto-albero di destra, non verrà mai individuata se non è stato già 
espanso tutto il sotto-albero di sinistra, senza aver trovato una soluzione. Quindi se individua 
una soluzione la restituisce indipendentemente dalla sua ottimalità. 

Si guadagna rispetto alla ricerca in ampiezza nella complessità spaziale. Infatti non bisogna 
memorizzare l'intero albero, ma solamente il cammino dalla radice alla foglia, ed i 
fratelli non espansi di ciascun nodo del cammino di profondità $m$: $O(b\cdot m)$. Se 
l'albero ha rami infiniti, allora la ricerca non termina. 

\subsubsection{Algoritmo di Ricerca in Profondità Limitata}

Nella ricerca in profondità limitata si impone un limite alla profondità massima, per impedire 
di proseguire all'infinito su uno stesso cammino. Un nodo viene espanso solo se la lunghezza 
del cammino dalla radice al nodo è minore del massimo stabilito. Se non viene trovata 
alcuna soluzione restituisce il valore speciale taglio se alcuni nodi non sono stati espansi, 
altrimenti fallisce. 

Si possono utilizzare conoscenze specifiche al problema per fissare questo limite. Se si 
lavora su di un grafo si potrebbe utilizzare il diametro del grafo come la profondità. Se non 
si sceglie un valore adeguato per questo limite allora l'algoritmo non funzionerà correttamente. 
L'algoritmo è completo, se la soluzione è ad una profondità minore della lunghezza $l$ imposta, 
mentre non è ottimale. La complessità temporale e spaziale è rispettivamente $O(b^l)$ e $O(b\cdot l)$. 
Risolve il problema della completezza, ma non risolve l'ottimale.  

\subsubsection{Algoritmo di Ricerca Iterative-Deepening Search}

Questo algoritmo risolve il problema dell'ottimalità sugli algoritmo di ricerca in profondità 
senza conoscere un limite adeguato. Evita il problema della scelta del limite provando 
iterativamente tutti i limiti possibili fino a quando non individua una soluzione:
\begin{minted}[mathescape, escapeinside=||]{python}
    function |\textcolor{red}{ITERATIVE-DEEPENING-SEARCH}|(problem) returns solution or failure

    |\textcolor{black}{for}| depth = 0 to |$ \infty $|   do
        if |\textcolor{red}{DEPTH-LIMITED-SEARCH}|(problem, depth) succeeds
        then return ist result
    end
\end{minted}

Questo approccio combina i benefici di una ricerca in ampiezza con i benefici di una ricerca 
in profondità, poiché per ogni profondità vengono analizzati tutti i nodi. 

Quindi questo algoritmo è ottimale e completo per le condizioni della ricerca in ampiezza. 
La complessità spaziale è $O(b\cdot d)$, quindi non è esponenziale. Mentre la complessità 
temporale è simile a quella a quella della ricerca in ampiezza. L'algoritmo viene richiamato 
ogni volta che si aumenta il limite, quindi i nodi a profondità minore vengono generati 
ogni volta che viene eseguito nuovamente l'algoritmo. Quindi vengono generati in totale $N$ nodi:
\begin{gather*}
    N=\displaystyle\sum_{i=1}^db^i\cdot(d+1-i)
\end{gather*}
I primi $b$ nodi a profondità 1 vengono generati $d$ volte, fino ai nodi al livello $d$ generati 
una sola volta. Questo algoritmo è quindi circa l'11\% meno efficiente rispetto alla ricerca in 
ampiezza. Ma non rappresenta un incremento considerevole rispetto alla ricerca in ampiezza, 
quindi è accettabile. 

\subsection{Problema della Ripetizione degli Stati: Graph-Search}

Il problema della ripetizioni degli stati può provocare gravi complicazioni nel processo 
di ricerca. Questo problema sorge sopratutto quando sono possibili azioni bidirezionali ed 
in questo caso è possibile che gli alberi di ricerca siano infiniti. Si vuole quindi evitare 
quando è possibile di ripetere gli stessi stati in più nodi dell'albero di ricerca. 

Questi stati ripetuti possono in certi casi rendere il problema irrisolvibile, è conveniente 
controllare se uno stato è replicato. Se un algoritmo arriva ad uno stesso stato attraverso 
due cammini differenti, allora ha individuato uno stato ripetuto e deve scartare uno di questi 
due cammini, per determinare quale scartare si sceglie generalmente l'ultimo cammino ottenuto. 
Si scarta anche se questo cammino è migliore del cammino precedente. Si utilizza questo 
approccio poiché negli algoritmi di ricerca euristica, sotto alcune condizioni, quando trova 
un percorso questo è ottimo, quindi non sorgono problemi nello scartare cammini che portano allo 
stesso stato. 

In altre implementazioni dove non è garantito che il primo percorso trovato sia il migliore, 
bisogna controllare quale dei due cammini presenti il costo migliore. 
Per evitare la ripetizione bisogna contenere gli stati già visitati in memoria, tramite 
un'altra struttura dati chiamata insieme esplorato o lista chiusa, contenente ogni nodo 
espanso. 

Si modifica l'algoritmo Tree Search nell'aggiunta alla frontiera per verificare la ripetizione 
degli stati:
\begin{minted}[mathescape, escapeinside=||]{python}
    function |\textcolor{red}{GRAPH-SEARCH}|(problem) returns a solution or failure

    close  <- empty set
    fringe <- |\textcolor{magenta}{MAKE-QUEUE}|(|\textcolor{magenta}{MAKE-NODE}|(|\textcolor{magenta}{INITIAL-STATE}|(problem)))
    loop do
        if |\textcolor{magenta}{EMPTY}|(fringe) then return failure
        node <- |\textcolor{magenta}{REMOVE-FRONT}|(fringe)
        if |\textcolor{magenta}{GOAL-TEST}|(problem, |\textcolor{magenta}{STATE}|(node)) then return |\textcolor{magenta}{SOLUTION}|(node)
        if |\textcolor{magenta}{STATE}|(node) not in close
            then |\textcolor{magenta}{ADD}|(close, node)
            child_list <- |\textcolor{magenta}{EXPAND}|(node, |\textcolor{magenta}{OPERATOR}|(problem))
            |\textcolor{black}{for}| child_node in child_list 
                if |\textcolor{magenta}{STATE}|(child_node) not in close then 
                    fringe <- |\textcolor{magenta}{QUEUING-FN}|(fringe, child_node)
    end
\end{minted}

Questo approccio si chiama Graph Search, dove prima di aggiungere un nodo alla frontiera, si controlla 
se il suo stato è già stato avvistato. Si suppone che il primo cammino che raggiunge nuo stato $s$ 
è il più conveniente. 
Questo algoritmo realizza un albero direttamente sul grafo dello spazio degli stati, poiché 
è presente al massimo una singola copia di ogni stato. La frontiera separa nel grafo dello 
spazio degli stati in due regioni, una esplorata, ed una da esplorare. In questo modo ogni 
cammino dallo stato iniziale ad uno stato inesplorato deve passare attraverso uno stato 
sulla frontiera. 

L'algoritmo scarta sempre il cammino appena trovato, se lo stato raggiunto è ripetuto, quindi 
potrebbe scartare un cammino corrispondente ad una soluzione migliore. Potrebbe quindi 
non essere un algoritmo ottimale. 

Inoltre l'uso della lista chiusa significa che la ricerca in profondità e quella ad 
approfondimento iterativo non richiedono requisiti spaziali lineari. 


\section{Algoritmi}

%% TODO algoritmi

\subsection{Problema di Ricerca Globale}


Nei problemi precedenti quando l'algoritmo risolutivo raggiungeva uno stato obiettivo, il cammino verso quello stato rappresenta una soluzione 
del problema. Tuttavia in alcuni problemi lo stato obiettivo contiene tutte l informazioni rilevanti per la soluzione, dove il cammino è 
irrilevante. Come esempio si consideri il problema dell otto regine, è indifferente il cammino attraverso gli stadi intermedi, solamente lo 
stato finale, la disposizione delle regine nello stato finale. 

Algoritmi di ricerca locale si utilizzano per risolvere questo tipi di problemi. In questi problemi è sempre presente uno spazio degli stati 
ed uno spazio degli stati aventi ciascuno una sua valutazione. Si può immaginare questi stadi su una superficie del territorio, uno spazio dove l'altezza 
di questo stato rappresenta la sua valutazione. L'algoritmo quindi itera su ognuno di questi stati per cercare quello di altezza maggiore, o minore, identificando 
quindi la soluzione al problema, indipendentemente dal cammino preso per raggiungerla. Questi punti di massimo rappresentano dei picchi, i cui punti adiacenti sono 
strettamente minori dello stato di massimo. Quindi l'algoritmo che parte da uno stato iniziale deve cercare un massimo globale in questo spazio, determinando quale sia 
tra i vari massimi locali ed i massimi locali piatto, e le ``spalle'' massimi locali ``piatti'', prima di un massimo globale. 

%% TODO aggiungere immagine esempio dei massimi

Questi algoritmi chiamati anche di miglioramento iterativo, si muovono sulla superficie cercando questi picchi, senza tenere traccia del cammino effettuato, tenendo 
solamente traccia dello stato attuale e dei suoi vicini o successori, gli stati immediatamente adiacenti. 
Bisogna formulare il problema in modo che l'algoritmo non rimanga bloccato tra due massimi locali. 

\subsubsection{Algoritmo di Hill-Climbing}

Questo algoritmo segue sempre le colline più ripide, si muove sempre verso l'alto nella direzione dei valori crescenti, e termina quando raggiunge uno stato per il quale 
si ha un picco che non ha vicino stati di valore maggiore. Tuttavia questo algoritmo può rimanere intrappolato su massimi globali. 

Non viene memorizzato lo stato corrente, solamente il valore attraverso nodi che contengono lo stato ed il suo valore: \verb|NODO=<STATO, VALORE|. 

Esistono diversi tipi di algoritmi di questo genere per evitare di rimanere bloccati su picchi locali, utilizzando diverse tecniche. 
%% todo lista diversi hill climbing

%% TODO aggiungere minted

% algoritmo steepest ascent
Dallo stato corrente si ricava il suo valore, in seguito comincia un ciclo che prende in considerazione tutti i successori e si sceglie come \verb|next| il nodo di 
valore più alto. Se tutti i nodi adiacenti hanno un valore minore di \verb|next| allora questo rappresenta la soluzione dell'algoritmo e l'algoritmo termina, tuttavia 
questo stato potrebbe corrispondere ad un massimo locale, invece se esiste uno stato adiacente di valore maggiore, questo diventa \verb|next| e si passa alla nuova 
iterazione. 

%% todo pseudocodice dell'algoritmo

% algoritmo random restart
L'algoritmo contiene una componente di ripartenza casuale, questo infatti conduce una serie di ricerche di Hill-Climbing partendo da stati generati casualmente. Questo 
algoritmo da un punto di vista teorico è completo, poiché con una serie infinita di ripartenze, sicuramente l'algoritmo visita tutti gli stati del sistema, trovando sicuramente 
la soluzione ottima del problema. 

%% todo pseudocodice random restart e funzionamento componente casuale

Il ciclo interno ad ogni iterazione genera un ottimo locale, e prova ad evitare ottimi locali effettuando una nuova ricerca da un nuovo stato scelto 
casualmente. 

% algoritmo stochastic 
L'algoritmo Stochastic Hill-Climbing si ottiene modificando la procedura normale dell'algoritmo. Invece di valutare tutti i vicini dallo stato corrente, l'algoritmo 
sceglie casualmente uno solo dei suoi successori da valutare per determinare se si tratta il successore, ed in caso diventa il nuovo stato corrente \verb|next|, questo 
viene accetta con una probabilità che dipende dalla differenza della valutazione tra i due punti: $\Delta E=$\verb|VALUE(current)-VALUE(next)|. 

%%todo pseudocodice 

Il nuovo stato viene scelto con una probabilità $p$, calcolata come:
\begin{equation}
  p=\displaystyle\frac{1}{1+e^{\Delta E/T}}
\end{equation}

In seguito dopo una serie di iterazioni l'algoritmo restituisce uno stato ottimo. L'algoritmo ha quindi un solo ciclo, e può scegliere un nuovo punto con una probabilità 
$p$, quindi anche di valore minore. Questa probabilità dipende da un parametro $T$ costante durante l'esecuzione dell'algoritmo. Se vale 1, la probabilità di 
accettazione è sostanzialmente pari al 100\%. 

%% todo tabella / andamento rispetto a T di p

All'aumentare del valore i $T$ la probabilità di accettazione tende al 50\%, diventa quindi sempre meno importante la differenza della valutazione tra i due punti, 
effettivamente comporta una ricerca casuale, mentre al diminuire di $T$, la procedura rappresenta un semplice algoritmo Hill-Climbing. 

In caso di stati di valore uguale, la probabilità è del 50\%, se il valore dello stato \verb|next| è minore, la probabilità diminuisce, mentre se il valore di \verb|next| 
è maggiore dello stato corrente, la probabilità aumenta. 

%% todo tabella / andamento rispetto a $\Delta E$

Bisogna trovare una ``link function'' tra l'intervallo $\Delta E/T$ e la probabilità $p$. 

%% todo img andamento link function

La caratteristica di poter scegliere come passo uno stato peggiore questo algoritmo potrebbe evitare massimi locali. 

% algoritmo di simulated annealing

Questo algoritmo prende il nome dall'analogia con il processo di metallurgia per temprare un materiale, questo processo infatti raggiungere uno stato di struttura 
cristallina ad energia minima. La differenza principale con l'algoritmo stocastico, è la possibilità di variare il valore di $T$ gradualmente durante l'esecuzione 
dell'algoritmo. Il valore di $T$ parte da un valore elevato, per poi diminuire nel tempo, come se fosse la temperatura durante un processo di temperatura. 

%% todo pseudocodice simulated annealing

%% todo rec cambiamento di funzione probabilità

In questo algoritmo la probabilità di accettazione di un certo nodo viene implementata in modo differente rispetto all'algoritmo stocastico normale, poiché nel simulated 
annealing si considera solamente un semiasse dell'ascissa, dato che in caso \verb|next| sia migliore non viene calcolata la probabilità di accettazione. Si accetta sempre 
uno stato di valore migliore, mentre l'accettazione di uno stato peggiore dipende da una probabilità. 

%% todo caratteristiche

Molte implementazioni dell'algoritmo seguono la stessa sequenza di passi. Si assegna la variabile \verb|T|, e si sceglie uno stato corrente casuale al primo passo. Si determina 
un successore e si ripete per un certo numero di cicli nel secondo, e come passo finale si diminuisce la temperatura e si ripete dal primo passo se la temperatura non ha 
raggiunto la temperatura minima. Quindi l'algoritmo termina solamente quando la temperatura ha raggiunto la temperatura minima. 

Le aree di applicazione di questo algoritmo sono molto vaste nell'informatica. Proposto negli anni '80 ha avuto un enorme successo, anche solo nella sua versione base. 

\section{Introduzione a Python}

Python è un linguaggio di programmazione vastamente utilizzato nell'area dell'intelligenza artificiale e nel machine learning. Recentemente è diventato il linguaggio di 
programmazione più diffuso al mondo. Python è un linguaggio general-purpose, ideato da Guido van Rossum nel 1989, a più alto livello del C, poiché gestisce automaticamente le 
più fondamentali operazioni. 

La versione di Python utilizzata nel corso è la versione 3, nell'ambiente Anaconda. Può essere avviato tramite un interprete. Alla creazione di una variabile non 
è necessario definirne il tipo, il nome identificativo è arbitrario e può contenere numeri, ma non cominciare con un numero, viene consigliato di utilizzare un carattere 
minuscolo come primo carattere del nome. Esistono 33 parole chiave, non utilizzabili come nomi di variabili. Si può assegnare un valore ad una variabile tramite l'operatore 
\verb|=|, senza specificarne il tipo. 

Esistono una serie di operatori aritmetici come \verb|+|, \verb|-|, \verb|*|, \verb|\|, \verb|**| e \verb|\\| (floor division). Una differenza tra Python 2 consiste nella gestione del resto, infatti in Python 2 
viene considerata solo la parte intera del resto. 

Si possono inserire dati tramite la funzione \verb|input|, e si può convertire in un tipo specifico come \verb|tipo(var)|. I commenti vengono 
realizzati tramite il carattere \#. 

In Python sono presenti tutti gli operatori booleani di C, in aggiunta sono presenti altri operatori \verb|is| ed \verb|is not|. 

%% todo add espressioni booleane

Sono presenti gli stessi operatori logici di C, e come in C un qualsiasi valore diverso da zero corrisponde al booleano \verb|true|. 

In Python per identificare funzioni o istruzioni condizionali non si usano parentesi, ma si indenta di quattro posizioni. Dopo la condizione dell'
istruzione condizionale vanno inserite dei due punti \verb|:|. 


Si possono gestire le eccezioni con il costrutto \verb|try| ed \verb|except|:
\begin{minted}{python}
  try:
    # corpo del try
  except:
    # corpo dell'except
\end{minted}

In Python sono integrate tantissime funzioni utili, per svolgere attività comuni, utilizzabili senza doverle definire, queste sono funzioni ``built-in''. 
Per invocare funzioni presenti in un certo modulo si utilizzata la notazione puntata \verb|nomeModulo.nomeFunzione()|. 


Dati gli algoritmi analizzati precedentemente, si nota la necessita di introdurre generatori di numero casuali. La maggior parte di generatori casuali, sono deterministici, 
ovvero dato lo stesso input, generano gli stessi numeri casuali. Si utilizzano quindi numeri pseudo-casuali, generati da un calcolo deterministico, ma non è quasi possibile 
distinguerli da numeri generati casualmente. In Python esiste il modulo \verb|random| contenente funzioni pertinenti alla generazione di numeri casuali. 

Per definire nuove funzioni si utilizza la parola chiave \verb|def|, specificando il nome, tra parentesi tonde gli argomenti ed i due punti, indentando di quattro posizioni 
per scrivere il corpo della funzione:
\begin{minted}{python}
  def nomeFunzione(nomeArgomenti):
    # corpo della funzione
  # resto del codice
\end{minted}

Dopo aver passato degli argomenti ad una funzione, questi vengono assegnati a delle variabili locali. Si può utilizzare anche una variabile come argomento. Inoltre 
tutte le aggiunte possibili alle funzioni built-in, si possono effettuare sulle funzioni definite dall'utente come \verb|_twice|, per ripetere due volte la funzione. 

Si dividono le funzioni in due tipi ``fruitful function'' e ``void function'',  le prime restituiscono un valore, le secondo non restituiscono valore. Le prime quindi 
vengono usate per assegnare o inizializzare variabili. Se si tenta di assegnare il risultato di una void function ad una variabile, viene ottenuto un valore chiamato ``None''. 
Questo valore ha un suo proprio tipo. 

Si possono realizzare cicli tramite il costrutto \verb|while| o \verb|for|, seguito da una condizione booleana e dai due punti \verb|:|. 
Si può interrompere il ciclo con \verb|break|, e si può saltare l'iterazione corrente con \verb|continue|. Quando bisogna iterare su una collezione, si può realizzare 
un ciclo for-each:
\begin{minted}{python}
  for elem in Array
    # corpo del ciclo
\end{minted}

\end{document}